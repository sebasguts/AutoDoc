\documentclass[margin=0cm,innermargin=0.7cm,blockverticalspace=0.5cm]{tikzposter}
\tikzposterlatexaffectionproofoff
 
 
 % Set colortheme
 % (default, anil, armin, edgar, emre, hanna, james, kai, lena, manuel,
 % martin, max, nicolas, pascal, peter, philipp, richard, roman, stefanie,
 % vinay)
\usecolortheme{max}

 % Change some colors
% \definecolor{framecolor}{named}{black}
% 
% \settitlebodystyle{rectangular}
% \setblocktitlestyle{rounded}
% \setblockbodystyle{shaded}
 
\title{AutoDoc quick reference}
\author{Sebastian Gutsche, Prof. Dr. Max Horn}
\institute{University of Kaiserslautern, Justus-Liebig-Universit\"at Gie\ss en}

 % Begin document
\begin{document}

 % Title block
\titleblock[]
 \begin{columns}
 % Set first column
\column{0.5}

\block{Intro}{
\texttt{AutoDoc} is a simple \textsc{Gap}-package to
create \textsc{GAPDoc} files for your documentation while
declaring and installing functions. It extract given info
from the declarations, so you only have to fill in the gaps.
}

\block{Documentation functions}{
\texttt{
\begin{itemize}
 \item DeclareCategoryWithDoc( arg : options )
 \item DeclareRepresentationWithDoc( arg : options )
 \item DeclareOperationWithDoc( arg : options )
 \item DeclareAttributeWithDoc( arg : options )
 \item DeclarePropertyWithDoc( arg : options )
 \item DeclareGlobalFunctionWithDoc( arg : options )
 \item DeclareGlobalVariableWithDoc( arg : options )
 \item InstallMethodWithDoc( arg : options )
\end{itemize}
}
This method declare, like Declare*( arg ) would do.
In addition, it specifies various information documenting the declared thing.
They can be used to generate \textsc{GAPDoc} documentation files by calling
\texttt{AutoDoc} in a suitable way. \\
The optional parameters are like follows:\\

\textbf{description: }
  This contains a descriptive text which is added to the generated documentation.
  It can either be a string or a list of strings. If it is a list of strings, then these
  strings are concatenated with a space between them.\\
\textbf{return\_value: }
  A string displayed as description of the return value.\\
\textbf{arguments: }
  An optional string which is displayed in the documentation as arguments list of the operation.\\
\textbf{chapter\_and\_section: }
  An optional argument which, if present, must be a list of two strings, naming the chapter
  and the section in which the generated documentation for the operation should be placed.
  There are no spaces allowed in this string, underscores will be converted to spaces in
  the header of the chapter or the section.\\
\textbf{group: }
  This must be a string and is used to group functions with the same group name together
  in the documentation. Their description will be concatenated, chapter and section info
  of the first element in the group will be used.\\
\textbf{label: }
  This string is used as label of the element in the documentation. If you want to make a
  reference to a specific entry, you need to set the label manually.
  Otherwise, this is not necessary.\\
  Please be careful to not give two entries the same description by giving two declarations with
  the same name the same label.\\
\textbf{function\_label: }
  This sets the label of the function to the string \texttt{function\_label}.
  It might be useful for reference purposes, also this string is displayed as argument
  of this method in the manual.
  This really sets the label of the function, not the label of the ManItem.
  Please see the \textsc{GAPDoc} manual for more infos on labels and references.
}

\block{Documentation entry functions}{
The following functions do the same as the Declare*WithDoc, but do not declare anything.
They can be used to simply create entries without creating anything.
\texttt{
\begin{itemize}
 \item CreateDocEntryForCategory\_WithOptions( arg : options )
 \item CreateDocEntryForRepresentation\_WithOptions( arg : options )
 \item CreateDocEntryForOperation\_WithOptions( arg : options )
 \item CreateDocEntryForAttribute\_WithOptions( arg : options )
 \item CreateDocEntryForProperty\_WithOptions( arg : options )
 \item CreateDocEntryForGlobalFunction\_WithOptions( arg : options )
 \item CreateDocEntryForGlobalVariable\_WithOptions( arg : options )
\end{itemize}
}
}

\block{Additional functions}
{
There are some additional functions.
\begin{itemize}
 \item \texttt{SetCurrentChapter( name ) }
 \item \texttt{SetCurrentSection( name ) }
 \item \texttt{ResetCurrentChapter( ) }
 \item \texttt{ResetCurrentSection( ) }
\end{itemize}
 These functions set or reset a current chapter or section, which will be applied to
 entries without a chapter\_info instead of the default one. Note that setting a section
 without a chapter does nothing, and reseting the chapter also resets the section.
\begin{itemize}
 \item \texttt{ WriteStringIntoDoc( description : chapter\_info ) }
\end{itemize}
Writes a string or a list of strings given as argument \texttt{description} into doc.
\texttt{chapter\_info} is optional, but without a current chapter set, this would cause an error.
Also it is possible here to only give the chapter, not the section. It will be written in the chapter at the current point,
i.e. after the last written section.
}

\column{0.5}

\block{The \texttt{AutoDoc} function}
{

ToDo: Give a short overview over the most important parameters,
then give a reference to the actual manual

}


\end{columns}

\end{document}

